\setcounter{chapter}{1} 

\chapter{Theory}
Presenter innholder i kapitellet.

\section{Study area and Data}

\textbf{Add image of map}

Data in the period of 2004-2018 was downloaded at a hourly resolution.

\subsection{ERA5}
\paragraph{General paragraph on \acrlong{ra}}

\subsubsection{Temperature}
\subsubsection{Surface pressure}
\subsubsection{Specific and Relative humidity}


\subsection{Satellite images METEOSAT }
\textbf{Make image showing the history of sate
llite images read the filenames and correlate MSG X with date and time.}

\begin{itemize}
    
    \item Why this satellite --  only one over Europe which is geostationary. Doing machine learning on satellites which is not in geosycronic orbids are difficult due to the temporal inhomogenity. 
    
    \item Where is the data downloaded. 
    \item Four generations of meteosat satelites. It exist older satelites from meteosat but the used a different sensor and they are not avalable online. 
    
\end{itemize}

\subsection{ReGridding satelite images to fit ERA5 platecarre grid.}

Husk at du regridder noe som allerede er usikker så det er ikke sikkert du tilfører noe usikkerhet. 

\textbf{Add equation for area.}

\textbf{Image : View from space arrow view from era5 - cartopy maps.}

\subsubsection{Estimating dlon and dlat based on matries of lat lon.}

\subsubsection{Explain overestimated the area}
Because of equation for area doensn't allow different d-lat dlon at each edge (should be trapesodial shape) (we don't have the information to calculate this either.)

Because of this overestimation we weight the pixels with 
\begin{equation}
     \frac{area \cdot ERA_a}{AREA_{allpixels} }
\end{equation}

